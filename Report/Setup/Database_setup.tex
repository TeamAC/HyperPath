%%%%%%%%%%%%%%%%%%%%%%%%%%%%%%%%%%%%%%%%%%%%%%%%%%%%%%%%%%%%%%%%%%%%%%%
% Project Name : HyperPath                                            %
% Project Home : https://github.com/TeamAC/HyperPath                  %
% Part         : Setting up the MySQL database                        %
% Author       : chedi                                                %
% Comments     :                                                      %
%                                                                     %
%%%%%%%%%%%%%%%%%%%%%%%%%%%%%%%%%%%%%%%%%%%%%%%%%%%%%%%%%%%%%%%%%%%%%%%

\section{Database setup}
The MySQL database has become the world's most popular open source database
because of its high performance, high reliability and ease of use.
\subsection{Installing the MySql server}
On Fedora the MySQL database installation is an easy process, you just have to:
\begin{itemize}
  \item Use yum to install both mysql command line tool and the server:
  {\Console yum install mysql mysql-server}
  \item Enable the MySQL: {\Console service:/sbin/chkconfig mysqld on}
  \item Start the MySQL server:{\Console /sbin/service mysqld start}
  \item Set the MySQL root password:{\Console mysqladmin -u root password
  'new-password'}\footnote{The quotes around the new password are required.}
\end{itemize}

Additionnaly you can install the \url{http://wb.mysql.com/}{mysql-workbench}
which is a visual database design tool developed by MySQL and the highly
anticipated successor application of the DBDesigner4 project.\footnote{You will
need the workbench if you would like to open the database diagram in the code
repository}:
\\\\
{\Console yum install mysql-workbench}
\\\\
\textbf{Note:} If you plan not using both the database and the GlassFish server
on the same machine, please adjuste the firewall rules of the database machine so
the 3306 port is enabled. In most cases, just issuing the command:
\\\\
{\Console iptables -A INPUT -i eth0 -p tcp -m tcp --dport 3306 -j ACCEPT}
\\\\
will do the tric, if notn you have to check the firewall rules.
\subsection{Creating the database schema}
