\section{Introduction to Android}\index{Android}
\subsection{History of Android}\index{Android!History}
\subsubsection{The Not-So-Distant Past}

\paragraph{}
Historically, developers, generally coding in low-level C or C++, have needed to understand the specific hardware they were coding for, generally a single device or possibly a range of devices from a single manufacturer. As hardware technology and mobile Internet access has advanced, this closed approach has become outmoded.

\paragraph{}
More recently, platforms like Symbian have been created to provide developers with a wider target audience. These systems have proven more successful in encouraging mobile developers to provide rich applications that better leverage the hardware available.

\paragraph{}
These platforms offer some access to the device hardware, but require the developer to write complex C/C++ code and make heavy use of proprietary APIs that are notoriously difficult to work with. This difficulty is amplified for applications that must work on different hardware implementations and those that make use of a particular hardware feature, like GPS.

\paragraph{}
In more recent years, the biggest advance in mobile phone development was the introduction of Java hosted MIDlets. MIDlets are executed on a Java virtual machine, a process that abstracts the underlying hardware and lets developers create applications that run on the wide variety of devices that supports the Java run time. Unfortunately, this convenience comes at the price of restricted access to the device hardware.

\paragraph{}
Google acquired the start-up company Android Inc. in 2005 to start the development of the Android Platform (see Figure 1-3). The key players at Android Inc. included Andy Rubin, Rich Miner, Nick Sears, and Chris White.

\paragraph{}
The Android SDK was first issued as an \emph{“early look”} release in November 2007. In September 2008, T-Mobile announced the availability of the T-Mobile G1, the first smart-phone based on the Android Platform. A few days after that, Google announced the availability of Android SDK Release Candidate 1.0. In October 2008, Google made the source code of the Android Platform available under Apache's open source license.

\subsubsection{The Future}

\paragraph{}
Android sits alongside a new wave of mobile operating systems designed for increasingly powerful mobile hardware. Windows Mobile, the Apple iPhone, and the Palm Pre now provide a richer, simplified development environment for mobile applications. However, unlike Android, they’re built on proprietary operating systems that in some cases prioritize native applications over those created by third parties, restrict communication among applications and native phone data, and restrict or control the distribution of third-party apps to their platforms.

\paragraph{}
Android offers new possibilities for mobile applications by offering an open development environment built on an open-source Linux kernel. Hardware access is available to all applications through a series of API libraries, and application interaction, while carefully controlled, is fully supported.

\subsection{Why Android?}\index{Android!Why Android?}
\paragraph{}
Android has the potential for removing the barriers to success in the development and sale of a new generation of mobile phone application software. Just as the the standardized PC and Macintosh platforms created markets for desktop and server software, Android, by providing a standard mobile phone application environment, will create a market for mobile applications and the opportunity for applications developers to profit from those applications.

For the purpose of this project, we have the constraint of using the Java language, which is the main development language for android. The free availability of the SDK and simulator and easy integration with the eclipse framework make android the perfect choice for our case.
