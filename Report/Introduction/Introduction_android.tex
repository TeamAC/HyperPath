\section{Introduction to Android}\index{Android}
\subsection{History of Android}\index{Android!History}
\paragraph{}
Historically, developers, generally coding in low-level C or C++, have needed to understand the specific hardware they were coding for, generally a single device or possibly a range of devices from a single manufacturer. As hardware technology and mobile Internet access has advanced, this closed approach has become outmoded.

\paragraph{}
In more recent years, the biggest advance in mobile phone development was the introduction of Java hosted MIDlets. MIDlets are executed on a Java virtual machine, a process that abstracts the underlying hardware and lets developers create applications that run on the wide variety of devices that supports the Java run time. Unfortunately, this convenience comes at the price of restricted access to the device hardware.

\paragraph{}
Google acquired the start-up company Android Inc. in 2005 to start the development of the Android Platform. The key players at Android Inc. The Android SDK was first issued as an \emph{“early look”} release in November 2007. In September 2008, T-Mobile announced the availability of the T-Mobile G1, the first smart-phone based on the Android Platform. A few days after that, Google announced the availability of Android SDK Release Candidate 1.0. In October 2008, Google made the source code of the Android Platform available under Apache's open source license.

\paragraph{}
Android sits alongside a new wave of mobile operating systems designed for increasingly powerful mobile hardware. Windows Mobile, the Apple iPhone, and the Palm Pre now provide a richer, simplified development environment for mobile applications. However, unlike Android, they're built on proprietary operating systems that in some cases prioritize native applications over those created by third parties, restrict communication among applications and native phone data, and restrict or control the distribution of third-party apps to their platforms.

\subsection{Why Android?}\index{Android!Why Android?}
\paragraph{}
Android has the potential for removing the barriers to success in the development and sale of a new generation of mobile phone application software. Just as the the standardized PC and Macintosh platforms created markets for desktop and server software.

Here are several Android platform development advantages you can get if you choose Android:

\begin{itemize}
\item \textbf{Affordability and customization.} The main Android platform development advantage of the Google Android OS it is open source, meaning that its source code is freely available to developers. And because the code is free, there are no licensing fees to deal with and Android app developers have more leeway in terms of developing apps.

\item \textbf{A low barrier to entry in the mobile application market.} Because the technology is open sourced, Google Android apps are less expensive to produce compared to other mobile operating systems like the IPhone OS and Windows Mobile. Quite often the primary costs for building apps for Android include development and testing expertise of developers, cost of test devices and royalty fees in case you want to distribute your app to third-party app stores like the Android Market.

\end{itemize}


    

    

    

    

