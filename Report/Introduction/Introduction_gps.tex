%%%%%%%%%%%%%%%%%%%%%%%%%%%%%%%%%%%%%%%%%%%%%%%%%%%%%%%%%%%%%%%%%%%%%%%
% Project  : HyperPath                                                %
% Project @: https://github.com/TeamAC/HyperPath                      %
% Part     : GlassFish introduction                                   %
% Author   : chedi (chedi.toueit@gmail.com)                           %
% Comments :                                                          %
%                                                                     %
%%%%%%%%%%%%%%%%%%%%%%%%%%%%%%%%%%%%%%%%%%%%%%%%%%%%%%%%%%%%%%%%%%%%%%%

\section{Geolocalization}
\subsection{Introduction}
\subsection{Location-based service}
A location-based service (LBS) is an information or entertainment service,
accessible with mobile devices through the mobile network and utilizing the
ability to make use of the geographical position of the mobile device.

LBS can be used in a variety of contexts, such as health, indoor object search,
entertainment, work, personal life, etc...
\cite{geolocation:wikipidea:geomarketing}

LBS include services to identify a location of a person or object, such as
discovering the nearest banking cash machine or the whereabouts of a friend or
employee. LBS include parcel tracking and vehicle tracking services. LBS can
include mobile commerce when taking the form of coupons or advertising directed
at customers based on their current location. They include personalized weather
services and even location-based games. They are an example of telecommunication
convergence.
\cite{geolocation:wikipidea:lbs}

\subsection{Automatic Position Identification}
Though location-based services don’t require devices that know where you are,
there’s no question such a feature enhances the experience. But how should the
device get that information? One way is for users to identify their location,
such as by selecting cross streets from a menu. That, however, requires a manual
step, which can be difficult if the device doesn’t have a good input method. It
also means the device doesn’t have a continu- ous, storable record of its
position, which rules out a whole class of services. Moreover, in some cases
users don’t know where they are, so asking them isn’t helpful.

\subsection{GPS}
There are two primary mechanisms for determining the location of a handheld
device:
\begin{itemize}
  \item Global positioning system (GPS)
  \item Network-based triangulation
\end{itemize}

The US military developed GPS to help it locate its equipment and people, but
the system quickly found significant civilian uses. GPS satellites orbit the
earth in a configuration that means a person anywhere on the planet with a clear
view of the sky above generally can track the signal from at least three of
them. Receiver devices determine their location by measuring arrival time from
each signal source.

In the recent past, price (especially the hardware cost) have been the main
constraint for many small companies and individuals and the field had not
retained the due attention and publicity. But, with the tsunami of mobile
devices and smart phones equipped with GPS sensor and network connectivity on
the market, the geo-localistion is back and the applications are more
proliferous than ever. That's where the first challenge reside:
\begin{itemize}
\item Using a mobile device, acquire GPS data and send them to the main server
for storage and analysis.
\item Store the GPS data in the device in case of missing network connectivity
and send them back to the server when restored.
\item receive GPS data from the server and use them to guide the target through
a specific path
\end{itemize}

Because data without interpretation is merely obscure sequence of numbers, any
decent geo-localization system must have a complete set of tool for the
reception, storage, analysis and custom business logic. That's why the proposed
system will be provided with back office for managing the following tasks:
\begin{itemize}
\item Receive and save the GPS data send by the mobile target in a persisting
storage system (database, file...)
\item Using third party map data, render a graphical representation of the
current target position and the past movements history of one or more target.
\item Based on the user data, compute paths and send them as GPS coordinate to
the mobile target.
\item Provide complete reporting schema using the data acquired via the mobile
target.
\end{itemize}