%%%%%%%%%%%%%%%%%%%%%%%%%%%%%%%%%%%%%%%%%%%%%%%%%%%%%%%%%%%%%%%%%%%%%%%
% Project Name : HyperPath                                            %
% Project Home : https://github.com/TeamAC/HyperPath                  %
% Part         : BIRT introduction section                            %
% Author       : chedi                                                %
% Comments     :                                                      %
%                                                                     %
%%%%%%%%%%%%%%%%%%%%%%%%%%%%%%%%%%%%%%%%%%%%%%%%%%%%%%%%%%%%%%%%%%%%%%%

\section{Introduction to BIRT}
\index{BIRT!Introduction}
\label{sec:BIRT_intro}
\subsection{Introduction to business intelligence}

To give it a formal definition I would say that business intelligence is any tool
or method that allows developers to take data or information, process it,
manipulate it, and associate it with related information and present it to
decision makers. As for a simplified definition, it's presenting information to
decision makers in a way that helps them make informed decisions.

\subsection{The Current State of the BI Market}
you can divide the major players in this field into two categories: commercial
offerings and open-source offerings. Each category has its own benefits and
drawbacks. With the commercial offerings, typically you have familiar names such
as Actuate and Business Objects,offering various tools aimed at different levels
of business. Some of these tools are large and enterprise reporting platforms
that have the ability to process,analyze, and reformat large quantities of data.
One of the drawbacks of commercial offerings is the large price associated with
them, both in terms of purchasing and in terms of running them.

Then, you have your open-source offerings. Currently there are three big names in
the open-source reporting realm: JasperReport, Pentaho, and BIRT. Two of these
projects, JasperReports and BIRT, are run by commercial companies who make their
money by doing professional services for these offerings to small scale and
private projects. Again, there are a number of pros and cons associated with
open-source solutions. With open-source, you have full access to the source code
of the platform you choose. This allows you to add in functionality, embed it
with your existing applications, and actively participate in a development
community that is oftentimes very large and around the world. There is little
initial cost to open-source software in terms of purchasing, as open-source is
free. The cons are that there is typically a cost associated with finding
individuals who are knowledgeable in open-source.

\subsection{What is BIRT ?}
BIRT (which stands for Business Intelligence and Reporting Tools) is actually a
development framework. Adding the word "Tools" to the title acronym is
appropriate; BIRT is in fact a collection of development tools and technologies,
used for report development, utilizing the BIRT framework. BIRT isn't necessarily
a product, but a series of core technologies that products and solutions are
built on top of, similar in fashion to the Eclipse framework.

\subsection{Why BIRT?}
As mentioned before, there are other open-source reporting platforms out there.So
what makes BIRT stand out over JasperReports or Pentaho? With JasperReports, in
reality, it comes down to tastes. I prefer the "What You See Is What You Get"
(WYSIWYG) designer of BIRT over JasperReports. I also like the fact that it does
not require compiling of reports to run, and that it is an official Eclipse
project. However, that is not to say Jasper is not without strengths of its own.
Jasper does do pixel-perfect rendering of reports, which is something that BIRT
does not do. Later versions of Jasper also support the Hibernate HQL language.