%%%%%%%%%%%%%%%%%%%%%%%%%%%%%%%%%%%%%%%%%%%%%%%%%%%%%%%%%%%%%%%%%%%%%%%
% Project Name : HyperPath                                            %
% Project Home : https://github.com/TeamAC/HyperPath                  %
% Part         : Project goals                                        %
% Author       : chedi                                                %
% Comments     :                                                      %
%                                                                     %
%%%%%%%%%%%%%%%%%%%%%%%%%%%%%%%%%%%%%%%%%%%%%%%%%%%%%%%%%%%%%%%%%%%%%%%

\section{Project purpose}
The ability for mobile devices to provide access to information place them
amongst the most rapidly growing technologies in the world. The increase of
mobile phones equipped with location sensing technology and enabled to elicit
user experiences presents a valuable opportunity.
\\
\\
Mobile web portals already allow users to tag, rate, and describe locations as
they visit them, in order to aide the discovery of unknown locations of interest.
Similarly, users location history can be mined in order to provide personalized
recommendations for social events as they occur in the surrounding areas.
\\
\\
tacking the fact that several inexorable trends are driving the development of
location based computing in account, we proposed the creation of a complete set
of tools to manage and consume data flows arround services and geolocation data
using a set of Open Source technologies as part of our journey in the Java world.

\subsection{Project definition}
This project aims to address the workflow surrounding the acquisition, analysis
and reporting of GPS data collected via a mobile device and exposing these data
to the final user using a custom back office interface for the administration
side and an android client for the customers one.
\\
\\
The consumer or client will get a mobile application enabling him to define
various services preferences and to subscribe to notification flows. The same
application will play the role of tracking device and will synchronise with the
central database.
\\
\\
For the administration part, a heavy client using the RCP technology will expose
an interface for adding new services and categories, managing them and creating
reports about the users activities and the buisiness popularity. Such reports
will be generated using the BIRT framework.

\subsection{Challenges}
