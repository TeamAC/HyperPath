%%%%%%%%%%%%%%%%%%%%%%%%%%%%%%%%%%%%%%%%%%%%%%%%%%%%%%%%%%%%%%%%%%%%%%%
% Project  : HyperPath                                                %
% Project @: https://github.com/TeamAC/HyperPath                      %
% Part     : GlassFish introduction                                   %
% Author   : chedi (chedi.toueit@gmail.com)                           %
% Comments :                                                          %
%                                                                     %
%%%%%%%%%%%%%%%%%%%%%%%%%%%%%%%%%%%%%%%%%%%%%%%%%%%%%%%%%%%%%%%%%%%%%%%

\section{Introduction to GlassFish}\index{GlassFish!Introduction}
\subsection{Whats a java application server?}
\paragraph{}
A web application is a dynamic extension of a web or application server. Web applications are of the following types:

\begin{description}
\item[Presentation-oriented:] A presentation-oriented web application generates interactive web pages containing various types of markup language (HTML, XHTML, XML, and so on) and dynamic content in response to requests. Development of presentation-oriented web applications is covered in Chapter 4, JavaServer Faces Technology through Chapter 9, Developing with JavaServer Faces Technology.

\item[Service-oriented:] A service-oriented web application implements the endpoint of a web service. Presentation-oriented applications are often clients of service-oriented web applications. 
\end{description}

\subsection{Why GlashFish}
\subsubsection{GlassFish advantages}\index{GlassFish!Advantages}
With so many options in Java EE application servers, why choose GlassFish? Besides the obvious advantage of GlassFish being available free of charge, it offers the following benefits:

\begin{description}
\item[Java EE reference implementation:] GlassFish is the Java EE reference implementation. This means that other application servers may use GlassFish to make sure their product complies with the specification. GlassFish could theoretically be used to debug other application servers.

\item[Supports latest versions of the Java EE specification:] As GlassFish is the reference Java EE specification, it tends to implement the latest specifications before any other application server in the market. As a matter of fact, at the time of writing, GlassFish is the only Java EE application server in the market that supports the complete Java EE 6 specification.
\end{description}

