%%%%%%%%%%%%%%%%%%%%%%%%%%%%%%%%%%%%%%%%%%%%%%%%%%%%%%%%%%%%%%%%%%%%%%%
% Project Name : HyperPath                                            %
% Project Home : https://github.com/TeamAC/HyperPath                  %
% Part         : MySQL introduction section                           %
% Author       : chedi                                                %
% Comments     :                                                      %
%                                                                     %
%%%%%%%%%%%%%%%%%%%%%%%%%%%%%%%%%%%%%%%%%%%%%%%%%%%%%%%%%%%%%%%%%%%%%%%

\section{MySQL}
\subsection{Introduction}\index{MySQL!Introduction}
In today's interconnected world, it's almost impossible to find a business that
doesn’t depend on information in some form or another. Be it marketing data,
financial movements, or operational statistics, businesses today live or die by
their ability to manage, massage, and filter information flow in order to achieve
a competitive advantage.
\\\\
More often than not, all this data finds a home in a business’ relational
database management system (RDBMS), a software tool that assists in organizing,
retrieving, and cross-referencing information. A large number of such systems are
currently available, and you’ve probably already heard of some of them: Oracle,
Sybase,Microsoft Access, and PostgreSQL are well-known names. These database
systems are powerful, feature-rich software applications, capable of organizing
and searching millions of records at high speeds; as such, they’re widely used by
businesses and government offices, often for mission-critical purposes.
\\\\
Recently, though, more and more attention has focused on a relatively new entrant
in this field: MySQL. MySQL is a high-performance, multithreaded, multiuser RDBMS
built around a client-server architecture. Over the last few years, this fast,
robust, and user-friendly database system has become the de facto choice for both
business and personal use, notably on account of its advanced suite of data
management tools, its friendly licensing policy, and its worldwide support
community of users and engineers. This introductory chapter will gently introduce
you to the world of MySQL by taking you on a whirlwind tour of MySQL’s history,
features, and technical architecture.

\subsection{Top Reasons to Use MySQL}\index{MySQL!Advantages}

\begin{description}
\item[Scalability and Flexibility :] \index{MySQL!Advantages!Scalability}
The MySQL database server provides the ultimate in scalability, sporting the
capacity to handle deeply embedded applications with a footprint of only 1MB to
running massive data warehouses holding terabytes of information.

\item[High Performance :]\index{MySQL!Advantages!Performance} A unique
storage-engine architecture allows database professionals to configure the MySQL
database server specifically for particular applications, with the end result
being amazing performance results. With high-speed load utilities, distinctive
memory caches, full text indexes, and other performance-enhancing mechanisms,
MySQL offers all the right ammunition for today's critical business systems.

\item[Web and Data Warehouse Strengths :]\index{MySQL!Advantages!Data
warehousing} MySQL is the de-facto standard for high-traffic web sites because of
its high-performance query engine, tremendously fast data insert capability, and
strong support for specialized web functions like fast full text searches. These
same strengths also apply to data warehousing environments where MySQL scales up
into the terabyte range for either single servers or scale-out architectures.

\item[Comprehensive Application Development
:]\index{MySQL!Advantages!Development} One of the reasons MySQL is the world's
most popular open source database is that it provides comprehensive support for
every application development need. Within the database, support can be found for
stored procedures, triggers, functions, views, cursors, ANSI-standard SQL, and
more. MySQL also provides connectors and drivers (ODBC, JDBC, etc.) that allow
all forms of applications to make use of MySQL as a preferred data management
server. It doesn't matter if it's PHP, Perl, Java, Visual Basic, or .NET, MySQL
offers application developers everything they need to be successful in building
database-driven information systems.

\item[Management Ease :]\index{MySQL!Advantages!Management} MySQL offers
exceptional quick-start capability with the average time from software download
to installation completion being less than fifteen minutes. This rule holds true
whether the platform is Microsoft Windows, Linux, Macintosh, or UNIX. Once
installed, self-management features like automatic space expansion, auto-restart,
and dynamic configuration changes take much of the burden off already overworked
database administrators. MySQL also provides a complete suite of graphical
management and migration tools that allow a DBA to manage, troubleshoot, and
control the operation of many MySQL servers from a single workstation.

\item[Open Source Freedom:]\index{MySQL!Advantages!Freedom}
Pure awesomeness !
\end{description}
